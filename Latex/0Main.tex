
\documentclass[12 pt ,a4 paper, openright, oneside]{book}

% Pacchetti %
\usepackage[italian]{babel} % Italiano
\usepackage[utf8]{inputenc}
\usepackage{csquotes}
\usepackage{hyperref}
\usepackage{subcaption}
\usepackage{algorithm}
\usepackage{algpseudocode}
\usepackage{hyphenat}
\usepackage[hypcap=true]{caption}
\usepackage{setspace}
\usepackage[hang,flushmargin,bottom]{footmisc}
\usepackage{todonotes} % Note Todo
\usepackage{soul} % strikethrough 
\usepackage{caption}
\usepackage{placeins}
\usepackage{amssymb,amsmath}
\usepackage[official]{eurosym}
\usepackage{frontespizio}
\usepackage{booktabs}
\usepackage{listings}
\renewcommand{\lstlistingname}{Codice}% Listing -> Codice
\usepackage{mdframed}
\def\checkmark{\tikz\fill[scale=0.4](0,.35) -- (.25,0) -- (1,.7) -- (.25,.15) -- cycle;}
\usepackage{eurosym}
% Tabelle %
\usepackage[normalem]{ulem}
\useunder{\uline}{\ul}{}
\usepackage{array,ragged2e}
\usepackage{colortbl}
\newcolumntype{P}[1]{>{\RaggedRight\arraybackslash}p{#1}}

% Immagini %
\usepackage{graphicx, wrapfig, float, rotating}
% \graphicspath{{images/}}

% Bibliografia %
\usepackage[backend=biber,style=numeric,sorting=none]{biblatex}
\addbibresource{bibliografia.bib}

\begin{document}

% Qui ci andrà il frontespizio

\begin{titlepage}
\begin{center}
    \begin{figure}
        \centering
        \includegraphics[width=0.35\textwidth]{Images/logo.jpg}
    \end{figure}
    \begin{LARGE}
        {
        \bf Universit\`a degli Studi di Pisa
        }
        \\
        \vskip 0.1cm
    \end{LARGE}
%    \rule{15cm}{0.03cm}

    \begin{Large}
    \textbf {Dipartimento di Informatica}\\
    \end{Large}

    \begin{Large}
    \vskip 0.3cm
    Corso di Laurea Triennale in Informatica
    \end{Large}

\vskip 2cm
\begin{LARGE}
{\bf Studio del Dust Attack: un attacco all'anonimato di Bitcoin}
\\[3mm]
% {\bf V 0.1 - ultima modifica 14/12/2021}
% \\[3mm]
% {\bf A QUALIT\`A GARANTITA}
% \\[3mm]
%{\bf }
%\\[2mm]
\end{LARGE}

\vskip 2cm

\begin{tabular}{lr}
\normalsize  \textit{Candidato}: &\normalsize \textit{Relatori/Relatrici}:
\\[4mm] \large Jacopo Raffi &\normalsize \hspace{1.5cm}Prof.ssa \large Laura Emilia Maria Ricci

% \\[4mm] \normalsize { }                &\hspace{1cm} \normalsize Correlatori:
\\[3mm] \large                 &\hspace{1.5cm} \normalsize
Prof. \large Damiano Di Francesco Maesa 

\end{tabular}

\vskip 2cm

\begin{large}
%\rule{15cm}{0.03cm}\\[2mm]
Anno Accademico 2021-2022
%\rule{15cm}{0.03cm}\\
\end{large}

\end{center}
\end{titlepage}

% .. fine frontespizio


\tableofcontents

% il capitolo inizia a pagina nuova. Usare il comando \input{} per aggiungere senza andare a pagina nuova
\definecolor{lightgray}{rgb}{.9,.9,.9}
\definecolor{darkgray}{rgb}{.4,.4,.4}
\definecolor{purple}{rgb}{0.65, 0.12, 0.82}

\lstdefinelanguage{Python}{
  keywords={True, False, catch, def, float, or, len, not, return, None, if, in, while, do, for, elif, else, break, lambda},
  keywordstyle=\color{blue}\bfseries,
  ndkeywords={class, export, boolean, throw, implements, import, this},
  ndkeywordstyle=\color{teal}\bfseries,
  identifierstyle=\color{black},
  sensitive=false,
  comment=[l]{#},
  morecomment=[s]{"""}{"""},
  commentstyle=\color{purple}\ttfamily,
  stringstyle=\color{red}\ttfamily,
  morestring=[b]',
  morestring=[b]"
}

\lstset{
   language=Python,
   backgroundcolor=\color{lightgray},
   extendedchars=true,
   basicstyle=\footnotesize\ttfamily,
   showstringspaces=false,
   showspaces=false,
   numbers=left,
   numberstyle=\footnotesize,
   numbersep=9pt,
   tabsize=2,
   breaklines=true,
   showtabs=false,
   captionpos=b
}

\setstretch{1.2}
\chapter*{Ringraziamenti} % L'asterisco permette di non indicizzare il capitolo (e quindi non gli da un numero)
\begin{flushright}
\itshape 
Ringrazio la mia famiglia, in particolare i miei genitori e mio fratello, per avermi sostenuto nei momenti più bui della mia vita\\
Ringrazio Sara non solo per avermi sostenuto nei miei momenti difficili ma anche per avermi corretto lo stile delle relazioni scritte durante questi tre anni di università\\ 
Vorrei inoltre ringraziare i miei amici universitari Simone, Giacomo e Dario per la loro compagnia, per il loro aiuto, per tutte le risate che abbiamo fatto insieme e in generale per avermi fatto trascorrere dei bellissimi momenti durante questi tre anni di università.\\
Infine ci tengo a ringraziare i miei due relatori, la prof.ssa Ricci e il prof. Di Francesco, il Dott. Loporchio e la prof.ssa Bernasconi per avermi permesso di svolgere la tesi con loro e per tutta l'immensa pazienza e disponibilità che hanno avuto nei miei confronti.
\end{flushright}


\chapter{Introduzione}
Se ripercorriamo la storia dell'uomo, possiamo osservare come i mezzi per lo scambio di beni tra le persone sia profondamente mutato nel corso del tempo, in relazione alla trasformazione culturale e tecnologica cui é andata incontro l’umanità: dal baratto siamo passati alle monete d’oro, fino ad arrivare, all’immaterialità di assegni o carte di credito. Il concetto di denaro stesso, nel suo significato più generale di mezzo per consentire lo scambio di valore, non é sfuggito alla diffusione che Internet ha avuto, portando alla nascita di una tipologia di valuta completamente nuova denominata \textbf{criptovaluta} o \textbf{moneta elettronica}. 

Negli ultimi anni si sono sviluppate vari tipi di criptovalute, ognuna con i propri protocolli e le proprie specifiche tecniche, però la prima moneta elettronica sviluppata e che ha acquisito maggior valore nel tempo é \textbf{Bitcoin}. 

Bitcoin nasce con l'intento di creare una moneta completamente libera da controlli di tipo governativo o bancario, consentendo agli utenti di effettuare transazioni attraverso una rete decentralizzata secondo il modello Peer-to-Peer. Quindi la validazione della correttezza delle transazioni non é in mano ad una banca o ad un ente di terze parti, ma é nelle mani di tutti gli utenti della rete. 

Un’altra parola chiave per comprendere il funzionamento di Bitcoin é l’anonimato: Bitcoin é considerato un sistema anonimo, ma sarebbe corretto utilizzare l’espressione \textbf{pseudo-anonimo}.

Ogni utente che partecipa alla rete Bitcoin infatti viene identificato non dal proprio nome o cognome, bensì da un address, che nulla lascia trasparire sulla reale identità dell'utente che vi sta dietro. Un utente inoltre può utilizzare address differenti ogni volta che esegue
una transazione, rendendo in pratica complicato capire che dietro ad un insieme di address ci sia in realtà il medesimo utente. Il fatto che ogni utente possa generare le proprie transazioni porta ad una serie di problemi che non sono presenti nei sistemi di scambio tradizionali. Il problema più noto viene definito ''Double Spending". Prevenire il double spending significa impedire che un utente possa spendere più volte lo stesso importo in transazioni differenti, questo problema non esiste nei sistemi tradizionali perché lo scambio di valore avviene tramite ente centralizzato che ha il controllo dei fondi degli utenti. 

La soluzione adottata da Bitcoin é quella di permettere agli utenti di conoscere le transazioni validate precedentemente, quindi l'intero storico delle transazioni é reso pubblico a chiunque.

Il libro contabile elettronico contenente tutte le transazioni é implementato tramite una struttura dati completamente pubblica e immutabile: la blockchain, all’interno della quale sono registrati tutti i movimenti di Bitcoin a partire dalla prima transazione, fino ai giorni nostri. Nella blockchain quindi é possibile osservare tutte le transazioni eseguite dai vari address di Bitcoin, questo porta a un serio problema legato alla privacy degli utenti. 

Nel corso degli anni sono stati sviluppati diversi attacchi in grado di violare l'anonimato di Bitcoin, la tipologia di attacchi più studiata é quella basata sull'analisi della blockchain. L'obiettivo di questi attacchi é la violazione dell'anonimato di un utente aggregando diversi address in cluster, ciascuno associato ad un singolo utente; questo avviene tramite l'analisi della blockchain e l'utilizzo di particolari regole euristiche.

Lo scopo di questa tesi é l'analisi di un attacco di de-anonimizazzione definito \textbf{Dust Attack}. Il Dust Attack basa la propria strategia sull'invio del dust, una piccola quantità di criptovaluta presente in un UTXO sotto i limiti minimi di scambio, per questo motivo é necessario aggregare un importo dust insieme ad altri importi. L'euristica su cui si basa il Dust Attack afferma che tutti gli address di input di una transazione appartengono allo stesso utente, l'obiettivo dell'attaccante quindi é inviare il dust affinché venga speso insieme ad altri input in modo da poter collegare tutti gli address e formare un unico cluster da associare ad un singolo utente. In questo modo non solo l'attaccante può tracciare l'attività di un utente ma se scopre l'identità del proprietario di uno di questi address automaticamente scopre che tutti gli address del cluster appartengono al medesimo utente. Dopo aver scoperto l'identità del proprietario é possibile eseguire elaborati attachi di phishing atti a rubare le chiavi private degli address così da poter rubare i fondi di quell'utente.

Le analisi vertono in particolare sull'efficacia che possa avere il Dust Attack, in particolare verranno mostrate delle statistiche atte a dimostrare se e quanto sia stato speso il dust e se questi importi abbiano permesso una qualche de-anonimizzazione, infine verranno descritti alcuni pattern che potrebbero rappresentare dei possibili Dust Attack.

Segue questa introduzione il capitolo 2, nel quale verrà descritta la tecnologia alla base di Bitcoin. L'attenzione sarà posta sulla blockchain, sul funzionamento delle transazioni e il relativo protocollo di pagamento, verrà spiegato in breve l'UTXO set e infine verrà descritto il problema relativo all'anonimato di Bitcoin mostrando alcune euristiche utilizzate per la de-anonimizzazione degli address.

Nel capitolo 3 verrà spiegato più nel dettaglio il significato di dust e i suoi possibili utilizzi. Verrà descritto in dettaglio il Dust Attack, spiegando il suo funzionamento, le sue conseguenze e le possibili contromisure.

Nel capitolo 4 invece verrà mostrata la formattazione dei dati e verranno descritti alcuni algoritmi realizzati per l'analisi dei dati.

Il capitolo 5 mostrerà le statistiche realizzate. Inizialmente verranno mostrati i risultati ottenuti dopo aver applicato vari filtraggi del dataset. Successivamente verranno analizzate le distribuzioni degli input e degli output dust, in particolare la distribuzione del dust generato negli anni. In seguito gli importi dust sono stati classificati per mostrare come siano stati spesi negli anni, in particolare per dimostrare che il dust abbia permesso una de-anonimizzazione degli address. Dopo aver mostrato come diverse transazioni abbiano più di due address di input diversi viene mostrata la distribuzione del numero di address diversi all'interno delle transazioni con almeno due address differenti. Infine verranno descritti alcuni pattern di possibili Dust Attack. 
\chapter{Background}
In questo capitolo verrà descritta la tecnologia alla base di Bitcoin. Particolare attenzione verrà data alla blockchain, la struttura dati che costituisce il libro contabile dove vengono registrate le transazioni, al funzionamento delle transazioni, agli indirizzi. Infine verrà esposta la problematica relativa all'anonimato degli utenti di Bitcoin. 
\section{Funzioni hash}
Il meccanismo utilizzato da Bitcoin per garatnire l'integrità e l'immutabilità della  blockchain sono le funzioni hash.

Una funzione hash $f$: X $\longrightarrow$ Y è una funzione matematica avente come dominio X e codominio Y, insiemi finiti tali che $|X| >> |Y |$. Tale funzione prende in input elementi di X di lunghezza qualsiasi e produce in output, a prescindere dalla lunghezza dell’input, stringhe binarie di dimensione fissa, i cosiddetti fingerprint, chiamati anche immagini hash o semplicemente hash.

Una proprietà fondamentale delle funzioni hash è relativa al tempo necessario al loro calcolo: devono essere calcolate efficientemente, ossia a fronte di un input di $m$ bit, la complessità computazionale per produrne il fingerprint deve essere $O(m)$, lineare o comunque polinomiale nei bit su cui è rappresentato l’input.
Vista la grande differenza di cardinalità tra i due insiemi X e Y, inevitabilmente alcuni input diversi della funzione hash avranno la stessa immagine; questo fenomeno è detto collisione: $x_1$ e $x_2$ $\in$ X, con $x_1 \neq x2$ , collidono se la loro immagine hash è la stessa f ($x_1$) = f ($x_2$).
In crittografia si usano alcune famiglie di funzioni hash molto particolari, dette funzioni hash one-way o funzioni hash crittografiche, le quali devono rispettare altre importanti proprietà oltre a quelle descritte sopra:
\begin{enumerate}
    \item \textbf{Proprietà di one-way}: dato y $\in$ Y, output della funzione f, deve essere computazionalmente difficile invertire la funzione, ossia trovare un x $\in$ X tale che f (x) = y. Il termine one-way significa proprio questo: una funzione hash ``facile” da calcolare, ovvero di complessità polinomiale rispetto al numero di bit dell’input, ma ``difficile” da invertire ovvero di complessità esponenziale, il che rende l’inversione praticamente inattuabile.
    \item \textbf{Proprietà di claw-free}: Per la funzione f, deve essere computazionalmente difficile determinare due elementi $x_1$ e $x_2 \in$ X, $x_1 \neq x_2$, tali che f ($x_1$) = f ($x_2$). Ciò significa che per una funzione hash crittografica non deve essere possibile trovare praticamente due elementi che collidono.
\end{enumerate}
\section{Bitcoin}
Bitcoin, l’unità monetaria elettronica a cui facciamo riferimento in questa tesi, è stata sviluppata da Satoshi Nakamoto, un misterioso autore giapponese la cui identità resta a tutt'oggi ignota, tanto da indurre molti a pensare che si tratti di uno pseudonimo, o che dietro a tale nome si celi in realtà non una singola persona, ma addirittura un gruppo di ricercatori o di informatici. L’articolo in cui viene presentato l’intero protocollo Bitcoin viene pubblicato nel 2008, sotto il nome di ”Bitcoin: A Peer-to-Peer Electronic Cash System” \cite{nakamoto2008bitcoin}; l'articolo contiene la descrizione dettagliata del protocollo alla base del funzionamento di Bitcoin.

La peculiarità di tale sistema è l’uso di una rete Peer-to-Peer utilizzata per effettuare, diffondere e validare le transazioni. L’intero storico delle transazioni viene mantenuto in un libro contabile distribuito e di pubblica consultazione. La grande e difficile sfida che Bitcoin dunque si pone è quella di coniugare l’anonimato degli utenti con un’alta affidabilità relativamente alle transazioni e alla loro validità e integrità.

A fronte della sfida tra trasparenza e affidabilità, è fondamentale definire un’implementazione del libro contabile che impedisca alterazioni di transazioni già registrate e validate: ricordiamo che in questo contesto paritario e distribuito, nessun controllo viene effettuato da parte di entità centrali, come per esempio le banche.


La soluzione ideata da Nakamoto per garantire l’integrità dello storico delle transazioni è stata quella di implementare il libro contabile tramite una particolare struttura dati: la blockchain. Come mostrato in figura \ref{fig:blockchain}, questa struttura si compone di una serie di blocchi collegati tra di loro come in una catena: ogni blocco racchiude un insieme di transazioni effettuate in un certo periodo temporale.

Il blocco corrente, non ancora inserito, contiene le ultime transazioni la cui legittimazione deve essere ancora approvata, mentre i blocchi precedenti, già agganciati alla catena, si riferiscono a transazioni già validate, e che possono essere considerate immutabili. Il meccanismo che garantisce la totale immutabilità della struttura, pena la sua completa invalidazione, è la crittografia.

\begin{figure}[h!]
    \centering
    \includegraphics[scale=0.6]{Images/blockChaining.pdf}
    \caption{Schema della Blockchain}
    \label{fig:blockchain}
\end{figure}
\FloatBarrier
\section{Blockchain}
Un generico blocco $B_i$ all’interno della blockchain contiene la sequenza di transazioni relative ad un certo periodo temporale (supponiamo che esse siano n: $T_1$ , $T_2$ , ... ,$T_n$ ) e un valore hash $h_{i-1}$ che identifica il blocco precedente nella catena, ed è l’output di una funzione hash crittografica.

È inoltre presente un campo detto nonce, che è il risultato dell’operazione di mining, ovvero del procedimento che porta all’aggiunta di un nuovo blocco alla blockchain. Sono presenti anche altri dati all’interno del blocco, ma al fine di descrivere il meccanismo crittografico che salvaguarda l’integrità della struttura riteniamo sufficiente questo livello di dettaglio.

La Figura \ref{fig:blocchi} mostra la struttura di un generico blocco $B_i$ all’interno della blockchain.
\begin{figure}[h!]
    \centering
    \includegraphics[scale=0.4, trim = 0cm 0cm 0cm 3cm, clip]{Images/blocco_singolo.pdf}
    \caption{Schema di un generico blocco}
    \label{fig:blocchi}
\end{figure}
\FloatBarrier
Come accennato sopra, ogni blocco ha un suo identificativo univoco, una sorta di impronta digitale personale, che è l’output di una opportuna funzione hash.

Un insieme di funzioni hash crittografiche ampiamente usate è quello delle SHA (Secure Hash Algorithm). Una di queste è la SHA-256, funzione hash crittografica che da input di dimensioni variabili produce un fingerprint della lunghezza fissa di 256 bit, ed è la funzione utilizzata per calcolare gli hash dei blocchi all’interno della blockchain di Bitcoin.

L’immagine hash del blocco $B_i$ è calcolata applicando la SHA-256 all’input formato dalla concatenazione delle transazioni contenute in $B_i$ con l'header del blocco, che astraiamo rappresentando i due campi più importanti ovvero il nonce e l'hash del blocco precedente, come riassunto dalla figura \ref{fig:sha-256}:
\begin{figure}[h!]
    \centering
    \includegraphics[scale=0.4, trim = 1cm 4cm 0cm 4cm, clip]{Images/blocchi_sha.pdf}
    \caption{Calcolo hash blocco $B_i$}
    \label{fig:sha-256}
\end{figure}
\FloatBarrier
Mentre le transazioni sono note nel blocco, così come è noto l’hash proveniente dal blocco precedente, il valore del nonce è incognito. L’attività di mining consiste nel risolvere un puzzle crittografico: trovare il valore del nonce tale che l’immagine hash prodotta dalla SHA-256 inizi esattamente con $t$ zeri, dove $t$ è un valore prefissato dal sistema, e variabile nel tempo.

La ricerca del nonce per la corretta aggiunta di un blocco è denominata \textbf{Proof of work}. L'unico modo per di trovare il nonce che produca un fingerprint che inizi con $t$ zeri, è quello di applicare ripetutamente la funzione SHA-256 a nonce via via diversi, finchè non si giunge ad un hash che soddisfa la proprietà desiderata. La risoluzione del problema è esponenziale in $t$, infatti la ricerca del nonce richiede tempo $O(2^t)$, ciò rende il problema di difficile risoluzione, in quanto è necessario far eseguire una serie di calcoli computazionalmente pesanti. Il valore di t viene aggiornato periodicamente dal sistema, in modo che la validazione di un blocco richieda sempre in media circa 10/15 minuti di tempo. Negli anni sono stati progettati calcolatori ottimizzati a livello hardware con lo scopo di calcolare il più rapidamente possibile la funzione SHA-256.

Questo sistema garantisce che i blocchi già inseriti non possano essere modificati retroattivamente: cambiando il contenuto di un blocco, ne cambierebbe anche il valore hash, e ciò implica il dover ricalcolare tutti i nonce dei blocchi successivi ad esso, perchè altrimenti le immagini hash non corrisponderebbero più tra blocchi consecutivi; dunque ciò che è stato scritto sulla blockchain è da considerarsi immutabile, a meno di rendere inconsistente tutta la struttura anche alterandone una sola transazione.

I miner, ossia i nodi della rete che dispongono di nodi di elaborazione abbastanza potenti da risolvere le Proof of Work, sono di fatto gli unici che hanno il diritto di aggiungere transazioni alla blockchain, e nel farlo consumano un gran quantitativo di energia e di risorse di calcolo; per questo, il primo miner che riesce ad agganciare correttamente un blocco alla blockchain, riceve un premio in bitcoin, nella forma di una transazione priva di input e con output l’address del miner: la \textbf{coinbase}, registrata anch’essa nella blockchain come una normale transazione. Il premio, di 25 bitcoin nel 2014, viene dimezzato approssimativamente ogni quattro anni per essere definitivamente azzerato nel 2140 quando il numero complessivo di bitcoin esistenti dovrebbe raggiungere 21 milioni.
\section{Transazioni e Address in Bitcoin}\label{Transazioni}
In questo  paragrafo verranno approfonditi i concetti stessi di transazione e di address, descrivendo con un grado di dettaglio funzionale ai nostri scopi il protocollo di pagamento e, anche in questo caso, la crittografia che ne è alla base, sia per la generazione degli address che per il protocollo di transazione.

I software wallet di bitcoin, per esempio Samurai Wallet, sono caricati sul PC o sullo smartphone di ogni utente A e generano anzitutto una coppia di chiavi privata-pubblica $kprv_A$, $kpub_A$ per un cifrario asimmetrico su curve ellittiche \cite{curve}.

La chiave privata $kprv_A$ è ovviamente nota solo ad A e, come vedremo in seguito, è utilizzata da A per firmare le transazioni che genera e diffonde sulla rete. La chiave pubblica $kpub_A$ è utilizzata per controllare la firma di A ed è anche impiegata come suo identificatore: a tale scopo viene trasformata attraverso applicazioni ripetute della funzione hash SHA-256 (immagine a 256 bit), per essere poi compressa via RIPEMD-160 in un’immagine di 160 bit in testa alla quale è aggiunta una speciale sequenza che indica che la stringa complessiva è di fatto un indirizzo bitcoin.
\begin{figure}[h!]
    \centering
    \includegraphics[scale=0.5, trim = 1cm 2cm 0cm 2cm, clip]{Images/address_gen.pdf}
    \caption{Generazione address in Bitcoin}
    \label{fig:sha-256_address}
\end{figure}
\FloatBarrier
Gli address in Bitcoin si presentano quindi come stringhe alfanumeriche di questo tipo: \textit{1BAFWQhH9pNkz3mZDQ1tWrtKkSHVCkc3fV}.

Come già osservato questo indirizzo non corrisponde a una “locazione” dell'utente ma è un modo per identificarlo per poter inviargli una transazione. L'insieme delle coppie chiave pubblica-privata è contenuto all'interno di un apposito wallet, ad ognuna di queste coppie è associata un indirizzo però nessuno è a conoscenza del fatto che questi indirizzi appartengano allo stesso wallet.

Questo è importante perchè nonostante un wallet possa avere più address nessuno è a conoscenza che questi address appartengano al medesimo wallet; è possibile tracciare l’attività dei singoli address ma non l’attività di un intero wallet, garantendo un maggiore anonimato. Oltre a creare le chiavi e gli indirizzi il software cliente permette di costruire le transazioni.

Una transazione in Bitcoin può essere riassunta come “il mittente A vuole inviare X bitcoin al ricevente B”, completata dalla firma digitale di A. Si noti che A e B sono indicati attraverso i loro indirizzi Bitcoin e la transazione viene inviata per broadcast a tutti gli utenti. Diversamente da ogni altra forma di transazione economica, il ricevente B non ha la garanzia che la transazione sia valida finché, come vedremo, essa non è convalidata dalla rete. 

In effetti B sarebbe in grado di verificare sia la firma di A che i fondi che A ha a disposizione, poiché la chiave pubblica di A è nota e tutte le transazioni eseguite sulla rete sono pubbliche, ma non ha la possibilità di verificare che A non abbia utilizzato gli stessi fondi in istanti immediatamente precedenti per pagamenti diversi. Formalmente una transazione, omessi i dettagli tecnici, è innescata secondo il seguente schema:
\begin{center}
\textbf{Protocollo Bitcoin}:
\begin{enumerate}
    \item L’utente A genera un messaggio m = ($adr_A$, X, $adr_B$) , dove $adr_A$ , $adr_B$ sono gli indirizzi Bitcoin di A e B, e X è la somma da trasferire da A a B.
    \item L’utente A calcola l’hash $h = SHA256(m)$ del messaggio e genera la firma $f = D(h, kprv_A)$ per m.
    \item La coppia $<m, f>$ viene diffusa in broadcast da A sulla rete.
    \item L’utente B attende che la rete convalidi la transazione prima di accettarla.
\end{enumerate}
\end{center}

La chiave privata è l’unico strumento valido per dimostrare la proprietà in bitcoin da parte di un utente A.
La perdita della chiave comporta la perdita della proprietà a causa dell’impossibilità di firmare transazioni, e il furto della chiave da parte di un truffatore che la usi per firmare al posto di A comporta anch’esso la perdita di proprietà.

Un address acquisisce valore in base a quanti bitcoin detiene; quando si usa un address per pagare bitcoin (come input di una transazione), il suo valore viene azzerato, non è possibile spendere solo parte dei bitcoin che costituiscono il valore dell’address. Se un address A paga X bitcoin ad un address B, dopo la transazione il valore di A sarà azzerato, mentre il valore di B aumenterà di X bitcoin. In caso di eventuale resto, nel wallet del pagante si creerà un \textbf{change address}, che acquisirà un valore pari al resto ricevuto.

Il change address può essere un address diverso da A, oppure A stesso, a discrezione dell’utente: per rafforzare l’anonimato è estremamente consigliabile da parte del pagante, al fine di ricevere i resti, utilizzare sempre address diversi da quelli usati per effettuare il pagamento.

Quindi, come mostrato in \ref{fig:transaction}, gli input di una transazione sono output precedenti di un'altra transazione. Gli output di una transazione possono essere in due stati:
\begin{enumerate}
    \item \textbf{speso}: output che compare come input di una transazione successiva;
    \item \textbf{non-speso}: output che non sono input di alcuna transazione.
\end{enumerate}
\begin{figure}[h!]
    \centering
    \includegraphics[scale=0.4, trim = 1cm 2cm 0cm 3cm, clip]{Images/Transactions-input-and-output-in-blockchain.jpg.pdf}
    \caption{Transazioni in Bitcoin}
    \label{fig:transaction}
\end{figure}
\FloatBarrier

Tutti gli output non-spesi vengono salvati in un unico insieme denominato UTXO(Unspent Transaction Output) set \cite{utxo}.

L'UTXO set è il sottoinsieme degli output delle transazioni Bitcoin che non sono stati spesi e che possono essere utilizzati come input di altre transazioni. Ogni volta che viene creata una nuova transazione, vengono eliminati output dall'UTXO set poichè passano dallo stato non-speso allo stato speso, inoltre vengono inseriti nuovi UTXO poichè ogni transazione genera almeno un output. Fondamentalmente, le transazioni consumano UTXO (nei loro input) e ne generano di nuovi (nei loro output). Poiché l'UTXO set contiene tutti gli output non spesi, memorizza tutte le informazioni richieste per convalidare una nuova transazione senza dover ispezionare l'intera blockchain. Come suggerisce già il nome, gli UTXO sono effettivamente output di Bitcoin e, come tali, sono costituiti da due parti: l'importo trasferito all'output e lo script che specifica le condizioni da soddisfare per spendere l'output. Uno script è essenzialmente un elenco di istruzioni registrate con ogni transazione che descrivono come la persona successiva che desidera spendere i Bitcoin trasferiti possa accedervi, nel capitolo successivo verrà descritto il sistema di script in Bitcoin.

Il vantaggio principale di utilizzare l'UTXO set è la sua dimensione molto più ridotta rispetto all'intero database delle transazioni (la blockchain), l'UTXO set può essere quindi conservato nella RAM, il che velocizza il controllo di validità delle transazioni. L'algoritmo \ref{alg:tx_btc} schematizza la politica di accettazione locale di una transazione. La transazione, accettata localmente eseguendo questo algoritmo può non essere globalmente validata; le transazioni vengono aggiunte alla blockchain quando vengono confermate a livello globale.
\begin{algorithm}
\begin{algorithmic}
\State $tx \gets$ ricevi transazione
\For{input $i$ in $tx$}
    \If{($i$ not in local UTXO $or$ firma non valida)} 
        \State scarta $tx$ e fermati
    \EndIf
\EndFor
\If{sum(inputs) $<$ sum(outputs)}
    \State scarta $tx$ e fermati
\EndIf
\For{input $i$ in $tx$}
    \State rimuovi $i$ dall'UTXO set locale
\EndFor
\State inoltra $tx$
\end{algorithmic}
\caption{Gestione transazione Bitcoin}
\label{alg:tx_btc}
\end{algorithm}
\FloatBarrier
In Bitcoin quindi non esiste un saldo memorizzato, il bilancio di un address infatti è calcolato dai Software Wallet. Il wallet calcola il saldo di un address scansionando la blockchain e sommando gli importi di tutti gli UTXO relativi a quel singolo address. Infatti se un address riceve diversi importi, questi non vengono aggregati ma vengono distribuiti all'interno dell'UTXO set. Il bilancio complessivo di un wallet invece non è altro che la somma dei bilanci dei singoli address appartenenti a quel wallet. 
Non è possibile conoscere il saldo complessivo di un utente proprio perchè non sanno quali address appartengano ad un determinato wallet, esistono tuttavia delle tecniche e delle euristiche che permettono non solo di conoscere gli address di un wallet ma consentono anche di conoscere le informazioni personali del proprietario.
\section{Anonimato in Bitcoin}\label{Anonimato}
L'anonimato in Bitcoin dovrebbe garantire l'impossibilità di collegare address differenti, ovvero capire che appartengano ad uno stesso utente, e di scoprire l'identità di un proprietario di un wallet.

Nel protocollo Bitcoin in realtà si parla di pseudo-anonimato, infatti ogni utente utlizza degli pseudonimi, inviando e spendendo bitcoin tramite un numero arbitrario di address che non contengono informazioni sulla sua identità; per questo Bitcoin non viene definito anonimo ma pseudo-anonimo; esistono infatti diversi attacchi che permettono di violare l'anonimato di Bitcoin \cite{de-anonimizzazione} \cite{de-anon2}.

Uno di questi attacchi si basa sull'analisi delle transazioni salvate sulla blockchain che, come ricordiamo, è pubblica e immutabile. Lo scopo di questo attacco è di aggregare address differenti in cluster, ciascuno rappresentante un singolo utente, utilizzando le informazioni memorizzate nella blockchain. Una volta ottenuti questi cluster, se è poi possibile utilizzare informazioni esterne per identificare anche un solo address del cluster è quindi possibile identificare il proprietario di un singolo cluster. 

Per formare questi cluster vengono utilizzate delle euristiche basate sul comportamento degli utenti di Bitcoin, queste regole quindi dipendono dal tempo e non sono assolute. 

Le due euristiche più utilizzate sono:
\begin{enumerate}
    \item \textbf{inputs multipli};
    \item \textbf{change address}.
\end{enumerate}

La prima euristica si può definire nel seguente modo:
\begin{center}
    ``\textit{In una transazione multi-input tutti gli address riferiti in input appartengono allo stesso utente}"
\end{center}

Infatti ogni input di una transazione deve essere firmato, questo significa che per firmare la transazione sono necessarie le chiavi private di ciascun input. Questo implica che il firmatario conosca tutte le chiavi private oppure sia il proprietario di tutti gli address di input. Questa prima semplice euristica è stata utilizzata con successo in passato ma non è più valida per tutte le transazioni. La prima contromisura proposta è stata CoinJoin \cite{coinjoin} nel 2013, l'idea principale è che un insieme di utenti si incontri e crei collettivamente una singola transazione utilizzando i loro address. In questo caso si potrebbe generare quindi un falso positivo della regola euristica appena enunciata, infatti gli address di input in questo caso non appartengono ad un unico utente.

La seconda euristica si può enunciare come segue: 
\begin{center}
    ``\textit{il change address appartiene allo stesso proprietario degli address di input}"
\end{center}

Il problema principale di questa regola è capire quali degli address di output sia il change address, escludendo il caso di ''self change", ovvero quando il change address è uno degli address di input. 

Un address $a$ viene definito change address se e solo se:
\begin{itemize}
    \item la transazione non è coinbase;
    \item è presente più di un output;
    \item negli address di output non è presente alcun address di input(niente ''self change");
    \item è la prima volta che $a$ appare nella blockchain e non ci sono altri address di output che soddisfano questa condizione.
\end{itemize}

L'attacco che verrà analizzato in questa tesi, denominato \textbf{Dust Attack}, basa la propria strategia sulla prima euristica, in particolare l'attaccante agisce in modo che le vittime generino transazioni multi-input così da poter formare un cluster di address appartenenti ad un unico utente.







\chapter{Il Dust Attack}
Tra i molteplici attacchi all'anonimato di Bitcoin, un attacco interessante è il Dust Attack. Il Dust Attack è una nota tecnica di de-anonimizzazione il cui obiettivo è capire che certi address appartengano al medesimo wallet.

Gli obiettivi principali di questa tesi sono quindi:
\begin{itemize}
\item Definire il Dust Attack, spiegandone le conseguenze e le possibili contromisure;
    \item Presentare delle statistiche generali su come venga speso il dust, in particolare mostrare se e quanto possa essere efficace un Dust Attack mostrando gli effetti del dust sulla de-anonimizzazione degli address e dei relativi wallet;
    \item Descrivere alcuni pattern interessanti che sono stati individuati e che potrebbero essere messi in relazione con il  Dust Attack.
\end{itemize}

Siccome il Dust Attack sfrutta l'utilizzo degli importi dust, è importante quindi spiegare cosa sia il dust e in generale quali possano essere i suoi possibili impieghi.
\section{Definizione del dust}
Nelle transazioni di Bitcoin generalmente l'importo totale di input non viene speso completamente in output ma parte di esso viene inviato ai miners come fee. Formalmente la fee di una transazione, raccolta dai miners come ricompensa per includere la transazione in un blocco, è definita come:
\begin{center}
    $fee = \sum_{i=0}^{\#inputs} importo(input_i) - \sum_{j=0}^{\#outputs} importo(output_i)$
\end{center}
La fee quindi non è altro che la differenza tra l'importo totale di input e l'importo totale di output, assume valori $\ge 0$. In generale la fee viene stabilita autonomamente dal creatore della transazione per incentivare i miners a validare la transazione stessa. 

Tuttavia, per convenzione, esiste una fee minima, denominata \textit{minimum relay fee}, che una transazione deve pagare affinchè un nodo inoltri quella transazione agli altri nodi della rete. Questo valore non è prefissato e quindi può variare nel tempo. 

Nel client ufficiale, Bitcoin Core \cite{btccore}, esiste una convenzione, definita ``dust limit", introdotto per le transazioni standard, cioè quelle transazioni che i nodi sono suggeriti accettare e reinviare.

Lo scopo è quello di considerare non-standard le transazioni che generano degli output dust proprio perchè costerebbe di più per il destinatario spendere l'importo dust rispetto al valore dell'output creato. Infatti un importo dust non può essere speso singolarmente, poichè $<$ della \textit{minimum relay fee}, perciò deve essere aggregato ad altri input.

Il limite attuale è di 546 satoshi \cite{BtcDev}. Quindi tutti gli importi $<$ 546 satoshi sono definiti \textbf{dust}.
\section{Possibili utilizzi del dust}
Gli usi del dust possono essere molteplici. In generale può essere utilizzato per effettuare degli stress test della rete Bitcoin, proprio perchè permette di generare tante transazioni con molti output ad un basso costo. In questo caso il costo è determinato principalmente dalla fee della transazione e non tanto dagli output. Se consideriamo che il valore di 1 satoshi, unità minima di Bitcoin, vale meno di un centesimo, è possibile generare una transazione con 1000 output dust, 1 satoshi ciascusno, al costo di qualche centesimo.

Esistono però alcuni casi interessanti di uso del dust, tra questi consideriamo il suo utilizzo nel servizio di gambling Satoshi Dice, la scrittura di dati arbitrari tramite lo script OP\_RETURN e il Dust Attack.
\subsection{Satoshi Dice}
Satoshi Dice è un noto ``gioco di scommesse basato su blockchain" nato nell'Aprile 2012 \cite{SD}. A differenza dei tradizionali software di gioco online, le scommesse con Satoshi Dice possono essere inviate senza accedere al sito Web né eseguire alcun software client. Per giocare, viene effettuata una transazione Bitcoin a uno degli address resi pubblici da Satoshi Dice, caratterizzato da probabilità di vincita e quindi di pagamenti diversi. Tutti gli address resi pubblici da Satoshi Dice sono vanity address.

Un vanity address in generale è un address Bitcoin con le stesse funzionalità di qualsiasi altro address ma presenta una stringa personalizzata contenente una parola o un messaggio comprensibili alle persone. Un esempio di vanity address è 1BoatSLRHtKNngkdXEeobR76b53LETtpyT, che contiene la parola Boat, barca tradotto in italiano. 

Tutti gli address di Satoshi Dice presentano il prefisso \textbf{1dice}, ogni address presenta probabilità diverse di vincere la scommessa, generalmente minore è la probabilità maggiore è il compenso che si ottiene da una vincita. Per esempio l'address 1dice1e6pdhLzzWQq7yMidf6j8eAg7pkY offre una probabilità dello 0,0015\% di vincere 64 000 volte la puntata originale.

Per determinare se una scommessa è vincente o perdente, Satoshi Dice genera un numero casuale, il quale viene assegnato al giocatore. Il servizio successivamente utilizza una combinazione dell'hash della transazione della scommessa ed esegue un hash SHA2 a 512 bit prendendo come input l'hash della transazione utilizzando un segreto sconosciuto al giocatore. I primi quattro byte dell'output diventano il numero che determina se il giocatore abbia vinto la scommessa.

Il servizio, dopo aver determinato le scommesse vincenti e quelle perdenti, invia una transazione in risposta con il pagamento al vincitore e restituisce un sinoglo satoshi ai giocatori perdenti per comunicare loro la perdita. Questo significa che Satoshi Dice utilizza il dust, 1 satoshi, per comunicare la perdita di una scommessa.
\subsection{Dust in OP RETURN}
Le transazioni in Bitcoin sono molto più complesse \cite{script} di come descritte nel capitolo precedente \ref{Transazioni}, non contengono solo valori come address e importo ma includono anche del codice, in particolare ogni transazione comprende anche degli script che descrivono quali siano le condizioni che devono essere verificate affinchè il destinatario dei bitcoin possa spenderli. 

Bitcoin utilizza un sistema di script Forth-like, basato su stack e non Turing-completo; questa scelta è intenzionale in quanto impedisce possibili cicli infiniti. 

Gli script \cite{opcode} in Bitcoin sono:
\begin{itemize}
    \item Senza stato: non esiste uno stato prima dell'esecuzione di uno script né viene salvato dopo l'esecuzione.
    \item Deterministici: uno script viene eseguito alla stessa maniera in ogni sistema.
    \item Semplici e compatti: le istruzioni, gli OP\_CODE, sono codificate in un singolo byte. In totale ci sono 75 istruzioni funzionanti e 15 disabilitate.
\end{itemize}
Gli OP\_CODE in Bitcoin comprendono diverse categorie:
\begin{itemize}
    \item aritmetica di base: OP\_ADD, OP\_SUB;
    \item controllo di flusso: OP\_IF, OP\_ELSE;
    \item logica bit a bit: OP\_EQUAL;
    \item gestione stack: OP\_DROP, OP\_SWAP;
    \item hashing: OP\_SHA1, OP\_SHA256;
    \item verifica firma o multifirma: OP\_CHECKMULTISIG.
\end{itemize}

Le transazioni Bitcoin non forniscono un campo dove si possono salvare dati arbitrari \cite{arbdata}. Tuttavia, gli utenti hanno ideato diversi modi creativi per codificare dati arbtrari nelle transazioni, in particolare memorizzando valori arbitrari negli output delle transazioni. Questi metodi però modificavano il protocollo rendendo inspendibili gli output così generati. Il problema è che questi output non erano facilmente distinguibili dagli altri quindi in nodi della rete dovevano salvarli nei loro UTXO.

L'UTXO(Unspent Transaction Output) è l'insieme degli output riscattabili di tutte le transazioni della blockchain. Per essere valida, una transazione deve utilizzare solo elementi dell'UTXO come input.

Poiché questo set, per motivi di efficienza, è solitamente memorizzato nella RAM \cite{utxo} questa pratica influiva negativamente sul consumo di memoria dei nodi \cite{stresstest}.

Per risolvere tale problema nel 2014 è stato reso standard il codice OP\_RETURN \cite{opreturnstandard} . Questo OP\_CODE era presente fin dalla nascita di Bitcoin, però era considerato uno script non-standard proprio perchè la scrittura di dati arbitrari sulla blockchain non è stata considerata fin dall'inizio una buona pratica. Questo codice permette di segnare un output della transazione come non valido, quindi successivamente non potrà essere speso.

Poichè gli output con questo OP\_CODE sono contrassegnati come non spendibili, possono essere rimossi dall'UTXO set. In questo modo OP\_RETURN risolve il problema del consumo di memoria spiegato precedentemente. Il limite iniziale per la memorizzazione dei dati con questo codice era pianficato per essere di 80 byte, ma inizialmente ne furono concessi 40 (versione 0.9.0).  La versione 0.11.0 \cite{v11} ha esteso il limite di dati a 80 byte e la versione 0.12.0 \cite{v12} fino a un massimo di 83 byte. Inoltre una transazione standard non può contenere più di un output contenenti OP\_RETURN.

Siccome per scrivere dati arbitrari sulla blockchain è comunque necessario generare una transazione con almeno un output risulta quindi molto più conveniente generare output con importi dust proprio per ridurre al minimo i costi della transazione.

Come riportato in \cite{OP_RETURN} esistono diversi protocolli che sfruttano OP\_RETURN. Di solito, un protocollo è identificato dai primi byte di metadati allegati all'OP RETURN, ma il numero esatto di byte può variare da protocollo a protocollo. 

Questi protocolli possono essere divisi in categorie tra le quali:
\begin{itemize}
    \item \textbf{Risorse}: protocolli che sfruttano l'immutabilità della blockchain per certificare la proprietà, lo scambio e, infine, il valore dei beni del mondo reale. I metadati nelle transazioni vengono utilizzati per specificare ad es. il valore del bene, il importo del bene trasferito, il nuovo proprietario, ecc;  
    \item \textbf{Atti notarili}: protocolli per la certificazione della proprietà di un documento. Un utente può pubblicare l'hash di un documento in una transazione, e in questo modo può provarne l'esistenza e l'integrità;
    \item \textbf{Arte digitale}: protocolli per la dichiarazione dei diritti di accesso e di copia di arti digitali, come ad es. foto o musica.
\end{itemize}

In generale salvare dati sulla blockchain di Bitcoin non è una buona pratica, infatti anche la documentazione ufficiale di Bitcoin sconsiglia questo utilizzo \footnote{Le note di rilascio di  Bitcoin Core versione 0.9.0 affermano che: “Storing arbitrary data in the blockchain is still a bad idea; it is less costly and far more efficient to store non-currency data elsewhere.”}.
\section{Il Dust Attack}\label{dstatt}
Come analizzato nel capitolo precedente \ref{Anonimato} esistono delle euristiche che possono essere sfruttate per effettuare determinati attacchi in grado di violare l'anonimato di Bitcoin. L'attacco che verrà approfondito in questa tesi viene denominato \textbf{Dust Attack}. 

L'euristica utilizzata da questo tipo di attacco è la regola degli input multipli:
\begin{center}
    ``\textit{In una transazione multi-input tutti gli address appartengono allo stesso utente}"
\end{center}
L'obiettivo del Dust Attack è la de-anonimizzazione del proprietario di un wallet, in particolare l'attaccante vuole scoprire quali address appartengano allo stesso wallet, così da ottenere un'importante informazione che può essere utilizzata per effettuare altri attacchi più elaborati e molto più pericolosi.

Il Dust Attack deriva il proprio nome dall'impiego del dust, infatti il dust, nelle transazioni standard, non è mai speso singolarmente ma viene sempre unito ad altri importi. L'attaccante quindi, come mostrato in figura \ref{fig:Dust_attack}, invia centinaia, o migliaia, di importi dust ad address diversi confidando che vengano spesi successivamente in una nuova transazione insieme ad altri address. Se ciò avviene l'attaccante riesce a collegare questi address e capire che appartengono, con una certa probabilità, allo stesso utente. Gli address a cui viene inviato il dust sono tutti non-nuovi, ovvero address già comparsi in precedenza sulla blockchain. Le transazioni che inviano del dust ad address nuovi  molto probabilmente non rappresentano un Dust Attack, proprio perchè risulterebbe alquanto inusuale de-anonimizzare un address mai visto fino a quel momento; molto probabilmente il proprietario di un address nuovo è lo stesso utente che genera quella determinata transazione poichè è l'unico a conoscenza di quel determinato address.

Il Dust Attack ad un primo sguardo potrebbe sembrare particolarmente costoso, proprio perchè invia tanti importi ad address differenti, ma se analizziamo il valore in euro di un singolo satoshi, si può osservare che il costo è relativo; soprattutto se l'obiettivo finale è il furto dei bitcoin delle vittime. Infatti abbiamo che:
\begin{center}
    1 satoshi = 0.00016 € (in data 16/11/2022) 
\end{center}

Questo significa che 1 singolo satoshi vale meno di 1 centesimo, quindi se un attaccante genera una transazione con 1000 output dust e manda a ciascun address 1 satoshi, in totale spende 16 centesimi, escludendo la fee.   
\begin{figure}[h!]
    \centering
    \includegraphics[scale=0.5, trim = 1cm 5cm 0cm 0cm, clip]{Images/dust_attack.pdf}
    \caption{Schema Dust Attack}
    \label{fig:Dust_attack}
\end{figure}
\FloatBarrier
Una volta effettuato l'attacco possono esserci due possibili esiti: 
    \begin{enumerate}
        \item attacco di successo;
        \item attacco fallito;
    \end{enumerate}
    
Nel primo caso, mostrato in figura \ref{fig:success}, la vittima genera una transazione in cui spende l'importo dust ricevuto insieme ad almeno un altro dei suoi address. 

Questi address quindi sono collegati tra loro ed è possibile capire che appartengano ad uno stesso utente. Se l'attaccante successivamente scopre le informazioni personali, come nome o email, del proprietario di uno di questi address automaticamente scopre che tutti gli altri address hanno lo stesso proprietario; inoltre è possibile tracciare l'attività di un singolo utente e non solo di un address singolo.
\begin{figure}[h!]
    \centering
    \includegraphics[scale=0.5,trim = 1cm 6cm 0cm 3cm, clip]{Images/successo.pdf}
    \caption{Schema di Dust Attack di Successo: l'address del dust viene speso con altri address}
    \label{fig:success}
\end{figure}
\FloatBarrier
Esistono invece due possibili motivi per cui un Dust Attack può fallire. In figura \ref{fig:fallito} vengono schematizzati questi due possibili esiti.

Nel primo schema la vittima spende il dust ricevuto, ma utilizza una transazione i cui input si riferiscono tutti all'address in cui è stato depositato il dust. In questo caso l'attaccante non ricava alcuna informazione utile perchè non riesce nell'intento di far unire address diversi di un utente nella medesima transazione. Nel secondo schema invece la vittima non spende il dust, troncando sul nascere l'attacco. 

Il Dust Attack in generale risulta più efficace soprattutto se il dust è diretto verso gli address che hanno un bilancio complessivo pari a zero proprio perché obbliga la vittima a spendere la cifra ottenuta con altri suoi address diversi.

\begin{figure}[h!]
    \centering
    \includegraphics[scale=0.5, trim = 1cm 6cm 0cm 3cm, clip]{Images/fallito2.pdf}
    (a)
    \includegraphics[scale=0.5, trim = 1cm 7cm 0cm 2cm, clip]{Images/fallito1.pdf}
    (b)
    \caption{Schema di Dust Attack Fallito}
    \label{fig:fallito}
    \subcaption{La vittima spende il dust con un solo address}
    \subcaption{La vittima non spende il dust}
\end{figure}
\FloatBarrier

È importante notare che lo scopo del Dust Attack non è quello di rubare fondi di altri utenti né quello di scoprire informazioni personali, per esempio nome e cognome, della vittima, ma permette di ricavare un'importante informazione, ovvero l'appartenenza di address differenti ad uno stesso utente, che può essere usata in seguito per effettuare attacchi elaborati e più pericolosi.

In generale il Dust Attack non necessariamente è legato a phishing o estorsioni ma potrebbe essere usato dalle autorità per tenere traccia degli utenti ed eventualmente rilevare attività illegali.

Infatti una volta otteuto un cluster di address riesco a tracciare l'attività di un singolo utente e non più di un singolo address. Le autorità potrebbero notare che certi utenti interagiscono con address legati a mercati neri, e quindi indagare ulteriormente per scoprire l'identità di queste persone. Uno dei punti fondamentali è legare informazioni personali, come e-mail, nome ed altro, ad un address Bitcoin. In molti casi sono gli utenti stessi che pubblicano sui forum i loro address, in altre situazioni invece è possibile sfruttare gli exchange come Coinbase.

Gli exchange sono servizi che permettono lo scambio tra criptovalute e valute tradizionali basandosi sul valore di mercato della criptovaluta. In exchange come Coinbase è necessario creare un account fornendo informazioni personali come nome, cognome, email ed altro e, una volta registrati, viene creato un wallet associato a quel particolare account. Poichè gli exchange sono esposti ad attacchi hacker molti utenti trasferiscono i loro bitcoin su address appartenenti a software wallet, per esempio Wasabi Wallet.

Una volta che un utente effettua un deposito dal wallet, vittima di Dust Attack, ad un account exchange ecco che l'attaccante collega gli address al proprietario. Una volta ottenuta l'identità del proprietario l'attaccante può eseguire elaborati attacchi di phishing oppure può estorcere denaro minacciandolo di rivelare a tutti l'informazione ottenuta. 

Il Dust Attack però non risulta particolarmente difficile da contrastare, nel paragrafo successivo verranno mostrati due metodi per difendersi da questo tipo di attacco.
\subsection{Contromisure al Dust Attack}
Due metodi che un utente che ha ricevuto un importo dust può utilizzare per contrastare il Dust Attack sono:
    \begin{enumerate}
        \item non spendere l'importo dust ricevuto; 
        \item utlizzare servizi di ``dust collecting". 
    \end{enumerate}
    
La prima soluzione, semplice ed efficace, permette di troncare l'attacco sul nascere. Infatti se il dust non viene speso l'attaccante non potrà mai collegare address diversi dello stesso utente. 

Diversi software wallet implementano questo metodo; per esempio Samurai Wallet \footnote{fonte:\url{https://twitter.com/samouraiwallet/status/1055345822076936192?lang=en}}, nel 2018, consigliò ai suoi utenti, possibili vittime di Dust Attack,  di contrassegnare come "do not spend" l'output dust ricevuto.

La seconda soluzione riguarda i servizi di ``dust collecting", per esempio Dust-B-Gone \cite{Dbg}.
Dust-B-Gone, ideato da Peter Todd già nel 2012, era un servizio che permetteva di disfarsi del dust, non lasciando il dust ricevuto come UTXO, ma trasformandolo in fee per i miner.

Il programma generava un'unica transazione alla quale potevano partecipare più utenti, ogni utente inseriva tra gli input della transazione l'importo dust ricevuto così da potersene liberare. Allo stesso tempo questo servizio proteggeva gli utenti da una possibile de-anonimizzazione, poichè address diversi di una transazione potevano appartenere ad utenti diversi. Un attaccante quindi non avrebbe mai potuto collegare questi address, o comunque ne avrebbe ricavato una falsa informazione.
Purtroppo questo servizio è stato chiuso nel 2016 per vari motivi tra cui il suo poco utilizzo \footnote{fonte: \url{https://twitter.com/SamouraiWallet/status/1293659938422652935}} . 
\chapter{Implementazione}
%Algoritmi e dati usati per l'analisi dei dati
In questo capitolo verranno mostrati il formato dei dati e gli algoritmi realizzati per le analisi.
Il codice è stato scritto in python, versione 3.9.12, distribuzione Anaconda.\\
La libreria per la realizzazione dei grafici è matplotlib versione 3.5.1 e la la libreria per le statistiche è Pandas versione 1.5.0\\
\section{Rappresentazione dei dati}
Il dataset su cui sono state condotte le analisi è stato ottenuto tramite un filtraggio della blockchain originale, il peso del file testuale contenente il dataset filtrato si aggira intorno ai 40 GB.\\
Ogni riga del file rappresenta una singola transazione che presenta il seguente formato:
\textbf{infos:inputs:outputs}.\\ Il campo \textbf{infos} contiene i seguenti dati:
\begin{itemize}
    \item timestamp: valore intero, rappresenta il tempo in cui è inserita la transazione;
    \item blockId: valore intero, l'ID del blocco dove è salvata la transazione;
    \item TxId: valore intero, l'ID della transazione, garantito essere univoco in quanto è un contatore, che vale 1 per la prima transazione e viene incrementato a seguire per le altre;
    \item fee: valore intero, indica il valore della fee pagata in quella transazione;
    \item approxSize: valore intero, indica il peso, in byte, approssimato della transazione.
\end{itemize}
Il campo \textbf{inputs} contiente gli input della transazione separati da '\textbf{;}'; possono esserci zero o più input.\\
Ogni input presenta la seguente struttura:
\begin{itemize}
    \item addrId: valore intero, rappresenta colui che sta pagando. Il valore numerico è stato assegnato in modo incrementale. L'addrId x è il primo che compare dopo l'addrId (x-1);
    \item amount: importo, in satoshi, speso dal pagante. Un satoshi equivale a $10^{-8}$ bitcoin; 
    \item prevTxId: l'ID della transazione in cui l’address pagante ha ricevuto l’importo che sta spendendo;
    \item offset: intero relativo alla posizione dell’address che sta pagando all’interno degli output della transazione specificata nel punto precedente, in cui il pagante ha ricevuto ciò che sta spendendo.
\end{itemize}
L'ultimo campo presenta una struttura pressoché uguale a quella del campo precedente.
Il campo \textbf{outputs} contiene gli output della transazione separati da '\textbf{;}'; deve esserci almeno un output.\\
La forma dei singoli output è la seguente:
\begin{itemize}
    \item addrId: intero con funzione analoga a quello degli input ma relativo a chi riceve l'importo; 
    \item amount: importo, in satoshi, che il ricevente incassa;
    \item script: valore intero, indica il tipo di script utilizzato per quell'importo; sono presenti 5 possibili valori: \\UNKNOWN=0; P2PK=1; P2PKH=2; P2SH=3; RETURN=4;\\ EMPTY=5;
\end{itemize}
La figura seguente riassume, su due righe invece che una per motivi di spazio, il metodo, appena enunciato, in cui sono memorizzate le transazioni nel file testuale:
\begin{mdframed}
timestamp,blockId,TxId,fee,approxSize:\\addrId,amount,prevTxId,offset[;input]:addrId,amount,script[;output]
\end{mdframed}
In seguito un esempio di una transazione presente nel file testuale:
\begin{mdframed}
1285666089,82560,121385,0,1000000,258:\\118901,9988099000,121384,0:118890,99,2;118902,9987098901,2
\end{mdframed}
La transazione di esempio con TxId 121385, timestamp 1285666089(martedì 28 Settembre 2010) presenta un solo input e due output: un solo addrId pagante, 118901, sta inviando
bitcoin a due address distinti: 118890 e 118902. L’address pagante
ha ricevuto i bitcoin che sta spendendo (9988099000 satoshi) nella transazione identificata dall' ID 121384; in essa era il primo output(offset zero). In questo caso, l’addrId 118890 ha ricevuto 99 satoshi, mentre l’addrId 118902 ne ha ricevuti 9987098901, entrambi gli output presentano script 2(script P2PKH). La fee complessiva, differenza tra l'importo totale di input e l'importo totale di output, è pari a zero, vuol dire che tutto l'input viene trasformato in output.\\\\
Gli input e gli output di ogni transazione, che presenta almeno un input o un output dust, sono stati successivamente salvati in due appositi file csv.\\
I file presentano la seguente forma:\\
Input:\\\\
\begin{tabular}{|r|r|r|r|r|r|r|}
\toprule
 timestamp &  blockId &   TxId &  addrId &     amount &  prevTxId &  offset \\
\bottomrule
\end{tabular}\\\\
Output:\\\\
\begin{tabular}{|r|r|r|r|r|r|}
\toprule
 timestamp &  blockId &   TxId &  addrId &     amount &  script \\
\bottomrule
\end{tabular}\\\\
Grazie a questa struttura è stato possibile classificare velocemente gli output in due gruppi distinti, tramite un'operazione stile SQL JOIN. Nel primo gruppo sono salvati gli output spesi, nel secondo quelli non spesi. Le due categorie presentano la medesima struttura:\\\\
\begin{tabular}{|r|r|r|r|r|r|r|r|r|}
\toprule
 tmp &  blockId &   TxId &  script &  addrId &  amount &  spentTxId &  spentBlock &  spentTmp\\
\bottomrule
\end{tabular}\\\\
I primi tre campi sono relativi alla transazione in cui è stato generato l'output mentre gli ultimi tre sono relativi alla transazione dove questo output appare come input, ovvero quando viene speso. AddrId è l'address che riceve e spende l'importo contenuto nel campo amount.\\\\
Un esempio di importo non speso è il seguente:\\
\begin{tabular}{|r|r|r|r|r|r|r|r|r|}
\toprule
1285666089 &    82560 & 121385 &  2 &      118890 &       99 &         -1 &          -1 &              -1 \\
\bottomrule
\end{tabular}\\\\
L'address 118890 ha ricevuto 99 satoshi nella transazione identificata dall' ID 121385 con timestamp 1285666089. Gli ultimi tre campi contengono il valore -1 per indicare che l'ammontare ricevuto non è stato speso. \\
L'esempio che segue mostra un caso di importo speso :\\
\begin{tabular}{|r|r|r|r|r|r|r|r|r|}
\toprule
1285666864 &    82561 & 121404 &  118890 &      99 &           2 &       124069 &       83232 &      1286048669\\
\bottomrule
\end{tabular}\\\\
In questo caso l'address 118890 ha ricevuto 99 satoshi nella transazione con ID 121404 e timestamp 1285666864, successivamente li ha spesi nella transazione con ID 124069 e timestamp 1286048669.\\
\section{Algoritmi}
Il primo passaggio è stato quello di filtrare le transazioni con almeno un importo dust, input o output. Le transazioni vengono poi salvate in un apposito file testuale  
\lstinputlisting{Codici/filter_dust.py}
Successivamente viene utilizzato ll seguente algoritmo per classificare le transazioni in due categorie e salvarle in due file distinti.
Nel primo file vengono salvate le transazioni che non hanno Satoshi Dice come input, nel secondo invece le transazioni generate da Satoshi Dice.
\lstinputlisting{Codici/filter_SD.py}
Gli output dust, con script diverso da OP\_RETURN, creati sono stati classificati in quattro categorie: dust speso con almeno un input proveniente da un altro address, dust speso con input provenienti dal medesimo address, dust speso in transazioni speciali, dust non speso. Tramite questo algoritmo, viene calcolato quanto siano presenti queste categorie nel tempo. Il parametro "tx\_sp" contiene gli identificativi delle transazioni dove tutto l'input viene trasformato in fee.
\lstinputlisting{Codici/temporal_dust.py}
Le transazioni, presenti nel dataset filtrato, che hanno almeno un input dust sono state classificate in tre categorie: transazioni con almeno due address diversi, transazioni con un solo address e transazioni speciali; anche in questo caso viene mostrato quanto esse siano presenti nel tempo.
\lstinputlisting{Codici/classification.py}
Per ognuna delle tre categorie di transazioni ho calcolato quante fossero OD e quante NOD. OD indica transazioni con soli input dust, mentre NOD indica transazioni con almeno un input non-dust.
\lstinputlisting{Codici/OD_NOD.py}
Successivamente ho calcolato, anno per anno, media e moda del numero di input, media e moda del numero di address diversi e la percentuale media di dust presenti negli input.
In seguito ho analizzato gli address che hanno ricevuto del dust per mostrare quale fosse il comportamento generale, concentrandomi in particolare su chi ha speso l'importo ricevuto.
\lstinputlisting{Codici/analisi_tx_success.py}
Passo successivo è stato l'analisi delle transazioni che hanno generato dust di successo, ovvero dust che è stato speso insieme ad altri address diversi. Ho suddiviso le transazioni in due categorie, quante presentano almeno un address nuovo e quante presentano solo address non nuovi. Delle transazioni che presentano address nuovi, ovvero address che compaiono per la prima volta on-chain, ho calcolato la percentuale media di address non nuovi.
\lstinputlisting{Codici/new_addresses_analisi.py}


\chapter{Analisi dati}
\section{Statistiche generali}
I dati analizzati comprendono le transazioni dal \textbf{3 Gennaio 2009} al \textbf{10 Agosto 2017}.

Come spiegato precedentemente il primo obiettivo è quello di presentare statistiche generali sull'uso del dust, mostrare gli effetti del dust sulla de-anonimizzazione di address e infine descrivere due pattern che potrebbero rappresentare dei Dust Attack.

Per questo motivo il primo compito è stato il filtraggio di tutte le transazioni dust, ovvero tutte quelle transazioni che comprendono tra gli input e/o tra gli output un importo $<$ 546 satoshi.
\begin{figure}[H]
\begin{mdframed}
 infos:inputs:118890,\textbf{99},2;118902,9987098901,2 \checkmark\\
 infos:21482214,984902,114569039,1;21482868,\textbf{1},73028796,240:outputs \checkmark\\
 infos:118925,9963398109,121409,0:118926,9962398010,2 \textbf{x}
\end{mdframed}
\caption{Esempi di transazioni accettate o rifiutate}
\label{tx_dust}
\end{figure}
\Floatbarrier
La figura \ref{tx_dust} mostra due esempi di transazioni dust e un esempio di transazione non-dust, le prime due transazioni vengono considerate nelle analisi successive proprio perchè contengono importi dust. 

La prima infatti contiene, tra gli output, un importo di 99 satoshi, la seconda invece contiene un importo di 1 satoshi tra gli input. L'ultima transazione invece è stata scartata proprio perchè tutti gli importi sono $\ge$ 546 satoshi. Le transazioni non-dust di qusto tipo sono state ignorate poichè le analisi vertono sulla generazione e sul consumo del dust e sulla de-anonimizzazione causata da questi importi.

Le transazioni totali sono 245 410 083 mentre le transazioni che generano o consumano dust sono  2 114 335, questo significa che il dust è presente solo nello 0.8\% delle transazioni totali; inoltre come riportato in \cite{dustAnalisi} 1 705 560 creano dust mentre solo 429 544 lo consumano. Da questi due ultimi risultati possiamo dedurre che ci sono transazioni in cui il dust è presente sia negli input che negli output.

Il passaggio successivo è stato il filtraggio delle transazioni generate da Satoshi Dice. Satoshi Dice, come spiegato in precedenza, è un noto servizio di gambling nato nell'Aprile 2012; questo servizio utilizza il dust per comunicare ai giocatori perdenti che hanno perso la loro scommessa. Data la notorietà del servizio risulta molto poco probabile che l'intento sia quello di un Dust Attack, inoltre le analisi vogliono mostrare il comportamento degli utenti nei confonti del dust proveniente da address sconosciuti; gli address di Satoshi Dice sono noti proprio perchè il servizio stesso li ha resi pubblici.

Le transazioni generate da Satoshi Dice sono 1 465 295, ovvero il 69\% delle transazioni dust. Questo risultato, oltre a dimostrare la popolarità del servizio, permette di concentrare le future analisi sul rimanente 31\%(649 040) delle transazioni.

Una volta ottenute tutte e sole le transazioni di interesse sono stati realizzati gli istogrammi mostrati in figura \ref{fig:dust_distribuzione}. Il primo istogramma mostra la distribuzione del numero di input dust, il secondo invece la distribuzione del numero di output. In entrambi i grafici non sono state contate le transazioni con 0 importi dust. La prima colonna infatti rappresenta, in entrambi gli istogrammi, l'intervallo [1, 50].
\begin{figure}[h!]
    \centering
    \includegraphics[scale=0.9]{Grafici/distribuzione_dust.pdf}
    \caption{Distribuzione numero input dust(sinistra) e output dust(destra)}
    \label{fig:dust_distribuzione}
    \subcaption{Intervalli di ampiezza 50}
    \subcaption{Primo intervallo [1, 50]}
\end{figure}
\FloatBarrier 
Entrambi gli istogrammi mostrano come siano molto frequenti le transazioni presenti nel primo intervallo di ampiezza [1, 50]. Nel primo grafico però notiamo anche che le transazioni con un elevato numero di input dust risultino poco usuali. Al contrario del primo grafico il secondo mostra come siano presenti tante transazioni che generano un elevato numero di output dust. Da questi due grafici inoltre possiamo già intuire che diversi output dust generati non vengano successivamente spesi.

Nelle analisi successive viene ignorato il dust generato con script OP\_RETURN. Questa scelta è motivata dal fatto che gli output con questo script non possono essere spesi, questa tipologia di dust quindi non può avere effetti sulla de-anonimizzazione; inoltre chi lo sta generando sicuramente non sta attuando un Dust Attack.

Il grafico \ref{fig:dust_created} mostra la generazione nel tempo del dust spendibile.
\begin{figure}[h!]
    \centering
    \includegraphics[scale=0.9]{Grafici/dust_created_year.pdf}
    \caption{Creazione dust nel tempo}
    \label{fig:dust_created}
\end{figure}
\FloatBarrier 
Dal grafico è possibile notare che già dal 2010 sono comparsi i primi output dust, anche se in quantità molto ridotta. Possiamo osservare una rapida salita tra il 2010 e il 2011, solo nel mese di luglio infatti sono stati generati 27 376 output dust.

Il picco della generazione di dust lo abbiamo nel 2013, dove sono state trovate transazioni legate "a catena"; nel paragrafo successivo verrà approfondito questo schema. Dopo il picco del 2013 osserviamo una diminuzione negli anni fino al 2017 dove sembrerebbe esserci una ricrescita. Bisogna però specificare che la maggior parte del dust generato nel 2017 proviene da due address: 1Enjoy1C4bYBr3tN4sMKxvvJDqG8NkdR4Z e 1SochiWwFFySPjQoi2biVftXn8NRPCSQC.

Questi due noti address sono comparsi per la prima volta nel 2014, in concomitanza con le olimpiadi di Sochi in Russia, generando transazioni con circa 750 output di 1 satoshi ciascuno. È importante notare che, nonostante il caos causato in vari forum Bitcoin, la tempesta di spam del 2014 "Enjoy Sochi" ha lasciato una piccola impronta sulla blockchain; solo 65 transazioni(48 750 output dust) sono state confermate in quell'anno.

L'aspetto singolare di questo fenomeno è che abbiamo visto echi di esso tornare negli anni successivi. Nel 2015 infatti sono state confermate 23 transazioni(1 725 output dust) mentre nel 2017 sono presenti 255 transazioni(191 250 output dust) generate da 1Sochi e 1Enjoy. Sebbene sia possibile che vengano eseguite dalla stessa entità, che spendeva importi da quegli address anche nel 2018, è anche possibile che queste transazioni fossero state generate nel 2014 e confermate solo nel 2015 e nel 2017. In questi tre anni il fenomeno "Enjoy Sochi" ha generato 343 transazioni per un totale di output dust intorno ai 255 000.

In generale sono stati generati 2 893 877 output dust con script diverso da OP\_RETURN; circa il 48.5\% è stato speso. Quindi non è presente una grande differenza tra la quantità di dust non speso(51.5\%) e la quantità di dust speso. 

La tabella \ref{tab:dust_spent_unspent} mostra le percentuali negli anni del dust speso e non-speso; il 2009 non è stato considerato perchè non è stato generato alcun output dust in quell'anno. La superiorità del dust non-speso rispetto al dust speso deriva quindi soprattutto dagli anni 2011 e 2017, anche se parte del dust del 2017 potrebbe essere stato speso successivamente al 10 Agosto 2017, data dell'ultima transazione del dataset. Il 2010, nonostante il dust non-speso sia l'83\%, non è molto rilevante poichè sono stati generati solo 14 output dust in totale. 
\begin{table}[H]
    \centering
    \begin{tabular}{|c|c|c|c|c|c|c|c|c|}
        \hline
           categoria/anno   & 2010 & 2011 & 2012 & 2013 & 2014 & 2015 & 2016 & 2017\\
        \hline 
         speso &  17\% & 0,2\% & 54\% & 44\% & 76\% & 94,6\% & 95,6\% & 10\% \\
         \hline
         non-speso & 83\% & 99,8\% & 46\% & 56\% & 24\% & 5,4\% & 4,4\% & 90\%  \\
         \hline
    \end{tabular}
    \caption{Dust Speso e Non-Speso negli anni}
    \label{tab:dust_spent_unspent}
\end{table}
Gli output dust sono stati suddivisi in quattro categorie:
\begin{enumerate}
    \item \textbf{Successo}: dust speso in transazioni che presentano almeno due address di input la cui fee è diversa dall'importo totale di input;
    \item \textbf{Fallimento}: dust speso in transazioni con un solo address di input;
    \item \textbf{Speciale}: dust speso in possibili transazioni di ``dust collecting", transazioni la cui fee è uguale all'importo totale di input;
    \item \textbf{Non speso}.
\end{enumerate}

I termini "Successo" e "Fallimento" si riferiscono alla possibilità di collegare gli address di input una volta che il dust è stato speso. Nonostante il ``dust-collecting" rappresenti un caso di de-anonimizzazione fallita, gli address di input infatti potrebbero appartenere ad utenti diversi, viene separata dalla categoria ``Fallimento" per mostrare se e quanto sia stato utilizzato questo servizio.

Il grafico \ref{fig:dust_year} mostra quanto siano presenti le categorie nel corso degli anni.
\begin{figure}[h!]
    \centering
    \includegraphics[scale=0.6]{Grafici/uso_del_dust_new.pdf}
    \caption{Uso del dust nel tempo}
    \label{fig:dust_year}
\end{figure}
\FloatBarrier

Non solo possiamo constatare graficamente quanto detto prima sulla differenza tra dust speso e non speso, ma possiamo anche trarre due importanti riflessioni. 

La prima riguarda l'uso del ``dust collecting", risulta evidente quanto poco sia stato utilizzato questo servizio. Il dust della categoria ``Speciale" non necessariamente è legato a Dust-B-Gone, questa categoria infatti è presente anche nel 2017 nonostante il servizio sia stato ufficialmente chiuso nel 2016. Se un utente decidesse di sua iniziativa di spendere tutto il dust in fee rientrerebbe comunque in questa categoria.

In generale infatti non è possibile distinguere una transazione di ``dust-collecting" da una transazione generata da un singolo utente, ma il numero di dust della categoria ''Speciale" costituisce un limite superiore al numero di dust spesi tramite Dust-B-Gone.

La seconda riflessione invece riguarda la tendenza a spendere il dust con almeno un altro address, in particolare notiamo un netto distacco nel 2015 dove la categoria ”Successo” rientra nell’ordine di $10^4$ mentre ”Fallimento” nell’ordine di $10^3$. In generale questa informazione è molto importante perchè dimostra come il dust possa essere efficace per la de-anonimizzazione di un wallet, ovvero capire che più address appartengano ad un medesimo utente.

Una volta constatata la presenza di transazioni con input dust e con almeno due address diversi, la fase seguente è l'analisi delle transazioni. 

Abbiamo tre categorie di transazioni: 
\begin{itemize}
    \item \textbf{2+ address}: transazioni che presentano almeno due address di input differenti, la fee però è diversa dall'importo totale di input;
    \item \textbf{1 address}: transazioni con un singolo address di input;
    \item \textbf{speciale}: transazioni dove l'importo totale di input è uguale alla fee.
\end{itemize}

In totale il dust è stato speso in 263 963 transazioni, la tabella \ref{tab:tx_categories} mostra la percentuale delle transazioni nelle tre categorie, la tabella \ref{tab:tx_categories_year} mostra una visione annuale.
\begin{table}[H]
    \centering
    \begin{tabular}{|c|c|}
        \hline
        2+ address & 63,2\%\\
        \hline
        1 address & 36,7\%\\
        \hline
        speciale & 0.1\%\\
        \hline
    \end{tabular}
    \caption{Transazioni nelle tre categorie}
    \label{tab:tx_categories}
\end{table}
\begin{table}[H]
    \centering
    \begin{tabular}{|l|c|c|c|c|c|c|c|c|}
        \hline
            categoria/anno  & 2010 & 2011 & 2012 & 2013 & 2014 & 2015 & 2016 & 2017\\
        \hline 
         2+ address(\%) & 100 & 93 & 55,6 & 68,4 & 55,9 & 64,5 & 62,7 & 75 \\
         \hline
         1 address(\%) & 0 & 7 & 44,4 & 31,5 & 44,09 & 35,4 & 37,2 & 24,7  \\
         \hline
         speciali(\%) & 0 & 0 & 0 & 0,1 & 0,01 & 0,1 & 0,1 & 0,3 \\
         \hline
    \end{tabular}
    \caption{Transazioni nel tempo}
    \label{tab:tx_categories_year}
\end{table}
Dalle tabelle \ref{tab:tx_categories} \ref{tab:tx_categories_year} confermiamo quanto detto prima, la prevalenza a spendere il dust in transazioni con almeno due address diversi e lo scarso utilizzo di servizi come Dust-B-Gone. Siccome sono presenti solo 17 transazioni della terza categoria 

La categoria "1 address" è stata ulteriormente suddivisa in due classi:
\begin{itemize}
    \item \textbf{NOD}: transazioni che presentanto input con importo $>$ 545  
    \item \textbf{OD}: transazioni con soli input dust.
\end{itemize}

Dalla tabella \ref{tab:OD_NOD_failed} possiamo dedurre che il fallimento nella de-anonimizzazione sia dovuto principalmente a quegli address che non sono vuoti, ovvero hanno ricevuto importi non-dust da altre fonti. Quindi questo permette di capire che il Dust Attack possa risultare più efficace contro gli address vuoti, con un bilancio complessivo tendente a zero.
\begin{table}[H]
    \centering
    \begin{tabular}{|c|c|c|}
        \hline
            NOD  & 92 712 & 95,5\%\\
        \hline 
            OD  & 4328 & 4,5 \%\\
        \hline
    \end{tabular}
    \caption{Classificazione OD e NOD}
    \label{tab:OD_NOD_failed}
\end{table}
Interessante invece è il secondo gruppo, in particolare due address appartenenti ad esso: 1JwSSubhmg6iPtRjtyqhUYYH7bZg3Lfy1T e 1PEDJAibfNetJzM289oXsW1qLAgjYDjLgN. Il fatto interessante del primo address è che che la sua chiave privata è stata compromessa \footnote{fonte:\url{https://privatekeys.pw/address/bitcoin/1JwSSubhmg6iPtRjtyqhUYYH7bZg3Lfy1T}}, questo consente a chiunque di riscattare bitcoin non appena gli vengono inviati.

Caso anomalo invece riguarda il secondo address, rinominato 1PED per semplicità. Questo address, come riportato in \cite{dustAnalisi} è un noto gambler di Satoshi Dice, la singolarità però deriva da alcune transazioni che ha generato. In queste transazioni infatti è presente un solo input di 1 satoshi e un solo output, ovviamente sempre di 1 satoshi. Questi singoli satoshi però non provengono da Satoshi Dice ma da altri address che seguono un pattern ben preciso, schematizzato nella figura \ref{fig:1PED}
\begin{figure}[h!]
    \centering
    \includegraphics[scale=0.4]{Images/1Ped.pdf}
    \caption{Schema transazioni legate a 1PED}
    \label{fig:1PED}
\end{figure}
\FloatBarrier
Ogni address invia due output, 1 satoshi con destinatario 1PED e un secondo importo che viene inviato ad un altro address che seguirà il medesimo schema; in tutte queste transazioni la fee è sempre di 50 000 satoshi. 1PED ha generato poi 1835 transazioni con 1 solo satoshi di input e 1 solo satoshi di output, la data è il 10 Marzo 2013 alle ore 16:20. È interessante sottolineare come 1PED sia riuscito a spendere un singolo satoshi senza la necessità di aggregare il dust con altri input. Questo significa che, nonostante queste transazioni fossero non-standard, il mineri ha comunque deciso di validarle, però sebbene questo comportamento sia anomalo risulta alquanto improbabile che si tratti di Dust Attack. Questa considerazione deriva dal fatto che tutti gli address di output delle transazioni generate da 1PED sono address nuovi.

Una volta esaminate le transazioni ``1 address" il prossimo passaggio è l'analisi delle transazioni della categoria "2+ address". Anche in questo caso vengono mostrati i gruppi OD e NOD. La tabella \ref{tab:OD_NOD_success} mostra come quasi sempre si riescano a collegare address vittima di dust con address più capienti.
\begin{table}[H]
    \centering
    \begin{tabular}{|c|c|c|}
        \hline
            NOD  & 166 778 & 99,9\%\\
        \hline 
            OD  & 128 & 0,1 \%\\
        \hline
    \end{tabular}
    \caption{Classificazione OD e NOD}
    \label{tab:OD_NOD_success}
\end{table}
È importante però capire quanto successo abbia avuto la de-anonimizzazione causata dal dust, la tabella \ref{tab:stat} mostra alcune statistiche generali di questa categoria. 
\begin{table}[H]
    \centering
    \begin{tabular}{|c|c|}
        \hline
            Percentuale media numero input dust & 29 \%\\
        \hline
            Moda percentuale numero input dust & 50 \%\\ %18 187 tx così
        \hline
            Media numero address diversi & 13\\
        \hline
            Moda numero address diversi & 2\\ %35011
        \hline
    \end{tabular}
    \caption{Statistiche generali}
    \label{tab:stat}
\end{table}

Possiamo subito osservare come moda e media tendano ad essere molto diverse tra loro, questo fatto permette di capire come le transazioni più frequenti non siano dominanti. In entrambe le situazioni infatti le transazioni che costituiscono la moda corrispondono a circa il 10\% nel primo caso(percentuale media numero di input dust) e al 20\% nell'ultimo(numero di address diversi). 

Il grafico \ref{fig:perc} mostra la distribuzione della percentuale di input dust mentre il grafico \ref{fig:diff_addr} mostra  la distribuzione del numero di address diversi.
 \begin{figure}[h!]
     \centering
     \includegraphics[scale=0.32]{Grafici/perc_dust.pdf}
     \caption{Distribuzione percentuale input dust}
     \label{fig:perc}
     \subcaption{Colonne di ampiezza 0.10}
     \subcaption{La prima colonna rappresenta l'intervallo [0, 0.10]}
 \end{figure}
\FloatBarrier
 \begin{figure}[h!]
     \centering
     \includegraphics[scale=0.32]{Grafici/num_addr.pdf}
     \caption{Distribuzione numero address differenti}
     \label{fig:diff_addr}
     \subcaption{Colonne di ampiezza 50}
     \subcaption{La prima colonna rappresenta l'intervallo [0, 50]}
 \end{figure}
\FloatBarrier
Dal primo istogramma notiamo la prevalenza degli input non-dust rispetto a quelli dust, la maggior parte della transazioni infatti ha una percentuale di input dust inferiore al 40\%. Questo dato è molto importante perchè mostra come il dust permetta di collegare soprattutto address più capienti, che potrebbero essere di maggiore interesse per un possibile attaccante. Il secondo istogramma invece mostra come siano presenti molte transazioni con un elevato numero di address diversi, permettendo un'efficace de-anonimizzazione dei relativi wallet.

È importante precisare che stessi address possano apparire in più transazioni, potrebbero quindi esserci delle de-anonimizzazioni ripetute. Infatti nonostante siano stati generati 2 893 877 output dust gli address a cui sono stati inviati sono 1 059 836 e solo il 29\%(312 114) lo ha speso. Di questo 29\% abbiamo che 259 252 hanno speso il dust in transazioni della categoria "2+ address" anche se 4 270 address hanno anche generato transazioni nella categoria "1 address". Una particolarità interessante delle transazioni che hanno generato il dust "Successo" è che in molti casi erano presenti tra gli output dust address nuovi, ovvero address che non erano mai comparsi sulla blockchain. 

Ci sono 98 198 transazioni che generano almeno un dust della categoria "Successo", circa il 59\%(58 146) non presenta address nuovi tra gli output. Nella fase finale delle analisi sono state prese queste transazioni così da trovare dei pattern interessanti di possibili Dust Attack di successo; nel paragrafo successivo verranno approfonditi due possibili schemi di Dust Attack.  
\section{Pattern Interessanti}
In generale è complicato affermare con certezza se un address stia compiendo un Dust Attack, però sono presenti due schemi, con certe proprietà, che potrebbero rappresentare un Dust Attack. La proprietà più importante riguarda l'assenza di address nuovi tra gli output dust, proprio perchè è alquanto improbabile tentare di de-anonimizzare un address mai visto fino a quel punto.

Il primo pattern, sintetizzato in \ref{fig:schema1}, risulta molto simile al caso 1PED. Abbiamo diversi address legati a catena che generano transazioni bi-output. Il secondo output, inviato al solito address, viene speso dall'attaccante per generare transazioni contenenti solo output dust. Inoltre l'attaccante, nell'immagine l'ovale rosso, non compare mai nella catena dei finanziatori, rappresentata dai cerchi azzurri.
\begin{figure}[h!]
    \centering
    \includegraphics[scale=0.4]{Images/dust_attack1.pdf}
    \caption{Primo Pattern}
    \label{fig:schema1}
\end{figure}
\FloatBarrier
Altre proprietà importanti di questo pattern sono: avere un unico finanziatore, generare sempre lo stesso numero di output dust, avere sempre il medesimo importo di input e infine l'attaccante esegue le transazioni in un breve intervallo di tempo. Questo schema è stato scoperto tramite l'address 1DiRy9Giiq1GCkAD7VMSrXoKVe2dimnovm, finanziato da 1Nj3AsYfhHC4zVv1HHH4FzsYWeZSeVC8vj. 

Inoltre sono presenti altri address che seguono lo stesso schema, pagati sempre dallo stesso finanziatore di 1Diry. Questo significa che il pattern appena enunciato può essere eseguito in parallelo utilizzano diversi address attaccanti. Potrebbero esserci in aggiunta più finanziatori che inviano denaro a più attaccanti. Per esempio questo address 16JLbXYe5xmxGNX8hiqooyTUJnhitNNqTh ha ricevuto fondi da 1NPfnbqZAMUnGuNuYwVZdCN4qVzUq4ejG4, l'aspetto interessante però è che il numero di output dust, l'importo dust inviato, l'importo input speso sono esattamente uguali a quelli usati da 1Diry; questo fatto fa ipotizzare che probabilmente questi address appartengano alla stessa persona. Ovviamente possono esserci eccezioni a questi due esempi trovati, per esempio avere più finanziatori o più attaccanti in una stessa catena, inoltre gli importi e il numero di output dust potrebbero avere una natura più randomica. In generale si potrebbero ipotizzare diverse variazioni di questo pattern, ovviamente ciò non implica che siano presenti.

Il secondo pattern, scoperto tramite 1JYvvL67LrSGCG77cy4rmpUXCFfSub4JkG (rinominato 1Jyvv), riguarda le transazioni "a catena" del 2013 citate nel paragrafo precedente. Ogni address della catena invia centinaia o migliaia di output dust e parte dell'importo, la fee è sempre $>$ 0, viene inviato al prossimo address che seguirà lo stesso schema. La singolarità di questo modello è che gli address della catena, ad eccezione del primo, vengono utilizzati solo per eseguire quell'unica transazione, non verranno mai più riutilizzati.
\begin{figure}[h!]
    \centering
    \includegraphics[scale=0.4]{Images/dust_attack2.pdf}
    \caption{Secondo Pattern}
    \label{fig:schema2}
\end{figure}
\FloatBarrier
Prendiamo come esempio l'address A3 della figura \ref{fig:schema2}. A3 compare in sole due transazioni. Nella prima, generata da A2, appare come destinatario di un importo non-dust, nella seconda invece spende questo importo inviando centinaia o migliaia di output dust e un importo non-dust con destinatario A4. Questo discorso però non si applica a colui che inizia questo pattern(A1), per esempio l'address 1Jyvv ha cominciato una serie di catene con questo modello.

Potrebbe essere interessante in futuro capire quanti address abbiano seguito schemi di questi tipo e se ci siano altri fini oltre a quello di un possibile Dust Attack.
\chapter{Conclusioni e sviluppi futuri}
In questa tesi abbiamo definito il Dust Attack, mostrando gli obiettivi dell'attacco, le conseguenze e le possibili contromisure. È stato analizzato il dust, esclusi gli importi dust generati da Satoshi Dice, in particolare è stato mostrato quanto dust come il dust abbia permesso la creazione di diversi cluster di address in varie transazioni. Infine sono stati descritti, non approfonditamente, due pattern che hanno generato un elevato numero di output dust inviato ad address non-nuovi. 

Alcuni sviluppi futuri di questo lavoro che possono essere suggeriti riguardano proprio questi due pattern \ref{pattern}. Potrebbero essere svolte ulteriori analisi per identificare quanti pattern di questo tipo siano presenti negli anni e se vengano utilizzati solo per eseguire un Dust Attack. Potrebbe essere interessante anche capire se gli address a cui viene inviato il dust siano scelti in modo casuale o se siano scelti address con determinate caratteristiche. 

Un'altra idea potrebbe essere il paragone tra il Dust Attack ed altri attacchi di de-anonimizzazione che non si basano sull'invio di dust. In particolare potrebbe essere interessante analizzare i cluster ottenuti mediante importi dust e i cluster ottenuti con altre tipologie di analisi della blockchain, così da capire se il Dust Attack abbia permesso la formazione di nuovi cluster mai visti fino a quel momento.

% Mostra bibliografia
\printbibliography[heading=bibintoc]

%``"
\end{document}
