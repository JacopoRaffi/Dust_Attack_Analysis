\chapter{Conclusioni e sviluppi futuri}
In questa tesi abbiamo definito il Dust Attack, mostrando gli obiettivi dell'attacco, le conseguenze e le possibili contromisure. È stato analizzato il dust, esclusi gli importi dust generati da Satoshi Dice, in particolare è stato mostrato quanto dust come il dust abbia permesso la creazione di diversi cluster di address in varie transazioni. Infine sono stati descritti, non approfonditamente, due pattern che hanno generato un elevato numero di output dust inviato ad address non-nuovi. 

Alcuni sviluppi futuri di questo lavoro che possono essere suggeriti riguardano proprio questi due pattern \ref{pattern}. Potrebbero essere svolte ulteriori analisi per identificare quanti pattern di questo tipo siano presenti negli anni e se vengano utilizzati solo per eseguire un Dust Attack. Potrebbe essere interessante anche capire se gli address a cui viene inviato il dust siano scelti in modo casuale o se siano scelti address con determinate caratteristiche. 

Un'altra idea potrebbe essere il paragone tra il Dust Attack ed altri attacchi di de-anonimizzazione che non si basano sull'invio di dust. In particolare potrebbe essere interessante analizzare i cluster ottenuti mediante importi dust e i cluster ottenuti con altre tipologie di analisi della blockchain, così da capire se il Dust Attack abbia permesso la formazione di nuovi cluster mai visti fino a quel momento.