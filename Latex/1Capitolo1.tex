\chapter*{Ringraziamenti} % L'asterisco permette di non indicizzare il capitolo (e quindi non gli da un numero)
\begin{flushright}
\itshape 
Fill me
\end{flushright}


\chapter{Introduzione}
Se ripercorriamo la storia dell'uomo, possiamo osservare come i mezzi per lo scambio di beni tra le persone sia profondamente mutato nel corso del tempo, in relazione alla trasformazione culturale e tecnologica cui è andata incontro l’umanità: dal baratto siamo passati alle monete d’oro, fino ad arrivare, all’immaterialità di assegni o carte di credito. Il concetto di denaro stesso, nel suo significato più generale di mezzo per consentire lo scambio di valore, non è sfuggito alla diffusione che Internet ha avuto, portando alla nascita di una tipologia di valuta completament nuova denominata \textbf{criptovaluta} o \textbf{moneta elettronica}. 

Negli ultimi anni si sono sviluppate vari tipi di criptovalute, ognuna con i propri protocolli e le proprie specifiche tecniche, però la prima moneta elettronica sviluppata e che ha acquisito maggior valore nel tempo è \textbf{Bitcoin}. 

Bitcoin nasce con l'intento di creare una moneta completamente libera da controlli di tipo governativo o bancario, consentendo agli utenti di effettuare transazioni attraverso una rete decentralizzata secondo il modello Peer-to-Peer. Quindi la validazione della correttezza delle transazioni non è in mano ad una banca o ad un ente di terze parti, ma è nelle mani di tutti gli utenti della rete. 

Un’altra parola chiave per comprendere il funzionamento di Bitcoin è l’anonimato: Bitcoin è considerato un sistema anonimo, ma sarebbe corretto utilizzare l’espressione \textbf{pseudo-anonimo}.

Ogni utente che partecipa alla rete Bitcoin viene identificato non dal proprio nome o cognome, bensì da un address, che nulla lascia trasparire sulla reale identità dell'utente che vi sta dietro. Un utente inoltre può utilizzare address differenti ogni volta che esegue
una transazione, rendendo in pratica complicato capire che dietro ad un insieme di address ci sia in realtà il medesimo utente. Il fatto che ogni utente possa generare le proprie transazioni porta ad una serie di problemi che non sono presenti nei sistemi di scambio tradizionali. Il problema più noto viene definito ''Double Spending". Prevenire il double spending significa impedire che un utente possa spendere più volte lo stesso importo in transazioni differenti, questo problema non esiste nei sistemi tradizionali perchè lo scambio di valore avviene tramite ente centralizzato che ha il controllo dei fondi degli utenti. 

La soluzione adottata da Bitcoin è quella di permettere agli utenti di conoscere le transazioni validate precedentemente, quindi l'intero storico delle transazioni è reso pubblico a chiunque.

Il libro contabile elettronico contenente tutte le transazioni è implementato tramite una struttura dati completamente pubblica e immutabile: la blockchain, all’interno della quale sono registrati tutti i movimenti di Bitcoin a partire dalla prima transazione, fino ai giorni nostri. Nella blockchain quindi è possibile osservare tutte le transazioni eseguite dai vari address di Bitcoin, questo porta a un serio problema legato alla privacy degli utenti. 

Nel corso degli anni sono stati sviluppati diversi attacchi in grado di violare l'anonimato di Bitcoin, la tipologia di attacchi più studiata è quella basata sull'analisi della blockchain. L'obiettivo di questi attacchi è la violazione dell'anonimato di un utente aggregando diversi address in cluster, ciascuno associato ad un singolo utente, ciò avviene tramite l'analisi della blockchain 